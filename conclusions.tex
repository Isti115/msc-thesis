\chapter{Conclusions}
\label{chp:conclusions}

Our formalization turned out to be adequate for proving theorems about the system, and while building proofs of correctness for more complicated programs is still a very tedious process, we have already managed to simplify it in several ways, which allowed us to verify partial properties and there might be even more opportunities available for making the implementation easier to work with that we have not noticed yet.

Find our source code at the \verb|tdk| branch of our GitHub repository: \url{https://github.com/Isti115/orsi-formalization/tree/tdk}.

\section{Further Work}

So far we feel like a good foundation has been laid down, but it is not nearly complete and has many ways, in which it can and should be continued. We discuss some of these in this section.

\subsection{Expanding the model}

The original material of the subject has only been partially implemented. There are multiple pending additions and changes, that have raised questions we do not have good solutions for yet, but with more research and experimentation they should be able to be integrated into the current system.

For example while the semantics of data channels are originally described using assignments as well, but their actual behavior is sort of implicitly more constrained, they cannot be assigned arbitrary values, their operations act more like methods on an object. The best solution for expressing these constraints is not clear yet, one possibility is to implemented them as separate instructions instead of expressions.

Another case, which the original material does not explicitly discuss is the out of bounds indexing of arrays. As this is basically undefined behavior, our current implementation handles it by indexing them with natural numbers and returning a default value when the index is higher than the length, but a cleaner solution could be achieved by indexing them with the appropriate finite set for their length. This would require changing the current representation of arrays:
\begin{code}
    \>[0]\AgdaFunction{evaluateType}\AgdaSpace{}%
    \AgdaSymbol{(}\AgdaInductiveConstructor{Array}\AgdaSpace{}%
    \AgdaBound{A}\AgdaSymbol{)}\AgdaSpace{}%
    \AgdaSymbol{=}\AgdaSpace{}%
    \AgdaDatatype{ℕ}\AgdaSpace{}%
    \AgdaOperator{\AgdaFunction{×}}\AgdaSpace{}%
    \AgdaSymbol{(}\AgdaDatatype{ℕ}\AgdaSpace{}%
    \AgdaSymbol{→}\AgdaSpace{}%
    \AgdaSymbol{(}\AgdaFunction{evaluateType}\AgdaSpace{}%
    \AgdaBound{A}\AgdaSymbol{))}\<%
\end{code}

A dependent pair would be sufficient, where the second function's domain is controlled by the first element representing the length:
\begin{code}
    \>[0]\AgdaFunction{evaluateType}\AgdaSpace{}%
    \AgdaSymbol{(}\AgdaInductiveConstructor{Array}\AgdaSpace{}%
    \AgdaBound{A}\AgdaSymbol{)}\AgdaSpace{}%
    \AgdaSymbol{=}\AgdaSpace{}%
    \AgdaRecord{Σ}\AgdaSpace{}%
    \AgdaFunction{ℕ}\AgdaSpace{}%
    \AgdaSymbol{(λ}\AgdaSpace{}%
    \AgdaBound{l}\AgdaSpace{}%
    \AgdaSymbol{→}\AgdaSpace{}%
    \AgdaSymbol{(}\AgdaDatatype{Fin}\AgdaSpace{}%
    \AgdaBound{l}\AgdaSpace{}%
    \AgdaSymbol{→}\AgdaSpace{}%
    \AgdaSymbol{(}\AgdaFunction{evaluateType}\AgdaSpace{}%
    \AgdaBound{A}\AgdaSymbol{)))}\<%
\end{code}

The problem raised by this change is that we would need to ensure that all indexing done during the execution of the program stays within the bounds of the arrays, which is hard to enforce, since indexes can not only be given by constants during the construction of a program, but also dynamically by evaluating expressions, over the results of which we have little to no control.

Some other constructions, like intersection of other programs, which we did not have enough time to address are still missing as well, these could be added and the theorems based on them could be proven.

\subsection{Optimization for education}

Since our long term goals include the potential introduction of this formalization as a tool for teaching the subject, we aim to create ways in which it can be suitable for such a purpose. A specific type of task that is taught in practical lessons and serves as an exercise during evaluations is a problem, where the students are given certain parametric statements, the truth of which are assumed, and from those they have to decide if the validity of another statement can be proven, and if so, they have to give a deduction.

The definition of such tasks can already be performed in our current system, for example by giving a function with the function signature describing the problem, with its parameters being the assumptions and its result type representing the statement in question. If the body of the function, and thus the deduction can be constructed, it means that the implication holds. This could be utilized for either providing opportunity for the students to practice on their own with the their computers providing them interactive help and being able to check the correctness of their own work, or if it is successfully introduced in classes, it may even be suitable for examinations with automatic grading.

The raw proofs for theorems and program properties are currently probably too complicated to be easily explained, but we have already started developing several helper functions that hide the lower level implementation and aim to provide a higher level interface, that could be used by students without having to understand all the underlying details.

\subsection{Automation}
Agda in itself already provides quite a good experience for interactively developing proofs using holes placed into the code and giving information about the types that are needed to fill them, but its auto solver is fairly limited to simple cases where the goal can easily be filled with an expression generated from the variables available in the close context. By expanding its search using the reflection capabilities of the language with methods specific to these tasks, more arguments could potentially be made implicit, and the verbosity of the otherwise sometimes inconveniently long proofs could be reduced.

\subsection{Generating executable code}
Another possible addition to the current system would be a way to export formal programs into a format that can either be further compiled or executed directly in an interpreter. This way it would be possible to experiment with the programs after proving their adherence to the given specification.