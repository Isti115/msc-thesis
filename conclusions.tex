\chapter{Conclusions}

Our formalisation turned out to be adequate for proving theorems about the system, and while so far it seems to be too complicated to build proofs of correctness for actual programs, we have already managed to simplify it in several ways, which allowed us to verify partial properties and there might be even more opportunities available for making the implementation easier to work with that we have not noticed yet.

Find our source code at \url{https://github.com/Isti115/orsi-formalisation}.

\section{Further Work}

A good foundation has been laid down, that has many ways, in which it can be continued.

\subsection{Expanding the model}

The original material of the subject has only been partially implemented, for example parts like the data channels and constructions (other than union), like intersection of other programs are still missing.

These could be added and the theorems based on them could be proven.

\subsection{Optimization for education}
% "Törtes feladatok" are already viable.
In the evaluations of the subject originally discussed there are tasks like proving a deduction rule, for which the code in its current form could already be utilized, for example by giving the function signature and 

