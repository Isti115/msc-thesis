%%%%%%%%%%%%%%%%%%%%%%%%%%%%%%%%%%%%%%
%% Header section of Latex document %%
%%%%%%%%%%%%%%%%%%%%%%%%%%%%%%%%%%%%%%

\documentclass[runningheads,a4paper]{report}
%% @author Daniel Lukacs, dlukacs@caesar.elte.hu, 2017

\usepackage[a4paper]{geometry}
\usepackage{t1enc}
\usepackage[utf8]{inputenc}
\usepackage{lmodern}

\usepackage[title,titletoc]{appendix}
\usepackage[section]{placeins}
\usepackage{relsize}

\usepackage[normalem]{ulem} %% Provides underlining.
\usepackage{caption} %% Provides captions.
\usepackage{mdframed} %% Provides frames around text and equations.
\usepackage{tikz-cd} %% Provides diagram drawing environment.
\usepackage{adjustbox} %% Provides additional tools to resize content.
% \usepackage[magyar]{babel} %% Provides foreign language support.

%%%%
%% Provides math related environments and directives.
\usepackage{amssymb}
\usepackage{amsthm}
\usepackage{amsmath}
\usepackage{latexsym}

%% See http://tex.stackexchange.com/questions/43835/conflict-between-amsthm-and-some-other-package
\let\proof\relax 
\let\endproof\relax

%%%%
%% Provides table environments and related directives.
\usepackage{array}
\usepackage{tabulary}
\usepackage{tabularx}
\usepackage{multirow}
\usepackage{hhline}

%%%%
%% Provides figure environments and related directives.
\usepackage{graphicx}
\makeatletter
\def\maxwidth#1{\ifdim\Gin@nat@width>#1 #1\else\Gin@nat@width\fi}
\def\maxheight#1{\ifdim\Gin@nat@height>#1 #1\else\Gin@nat@height\fi}
\makeatother

\usepackage{fancyvrb}
\usepackage{rotating}

%%%%
%% Provides environment to display source code.
\usepackage{listings} 
\lstset{ 
    literate=%
        {á}{{\'a}}1
        {é}{{\'e}}1
        {í}{{\'i}}1
        {ó}{{\'o}}1
        {ö}{{\"o}}1
        {ő}{{\H{o}}}1
        {ú}{{\'u}}1
        {ü}{{\"u}}1
        {ű}{{\H{u}}}1
        {Á}{{\'A}}1
        {É}{{\'E}}1
        {Í}{{\'I}}1
        {Ó}{{\'O}}1
        {Ö}{{\"O}}1
        {Ő}{{\H{O}}}1
        {Ú}{{\'U}}1
        {Ü}{{\"U}}1
        {Ű}{{\H{U}}}1
    } %% Customization of listings env., to enable non-English accents.

% \lstset{
%   frame=single,
%   basicstyle=\small,
%   language=Erlang,
%   numbers=left,
%   firstnumber=1,
%   numberfirstline=true,
% %  basicstyle=\ttfamily,
% %  columns=fullflexible,
% %   keepspaces=true,
% } %% Customization of listings environment


%%%%
%% Provides environment to display pseudocode.
\usepackage{algorithm}% http://ctan.org/pkg/algorithms
\usepackage{algpseudocode}% http://ctan.org/pkg/algorithmicx

\newcommand{\repeatcaption}[2]{%
  \addtocounter{figure}{-1}%
  \renewcommand{\thefigure}{\ref{#1}}%
  \captionsetup{list=no, labelformat=simple, labelsep=colon}%
  \captionof{figure}{#2}%
} %% Customization: Using the same figure twice with no new number. See http://tex.stackexchange.com/a/200229

%%%%
%% Provides directives to display followable URL references.
\usepackage{url}
\usepackage{hyperref}
\hypersetup{
  hidelinks,
  linkbordercolor = {0 0 1},
}

%% Customization: Followable links to appendix references.
\makeatletter
\appto{\appendices}{\def\Hy@chapapp{Appendix}}
\makeatother


%%%%
%% Custom document formatting.

% \renewcommand{\abstract}{ \begin{center}\textbf{Abstract}\end{center}}

\setcounter{tocdepth}{2}

\setlength{\parskip}{\baselineskip}%
\setlength{\parindent}{0pt}%

\makeatletter
\renewcommand\subsubsection{\@startsection{subsubsection}{3}{\z@}%
                       {-18\p@ \@plus -4\p@ \@minus -4\p@}%
                       {4\p@ \@plus 2\p@ \@minus 2\p@}%
                       {\normalfont\normalsize\bfseries\boldmath
                        \rightskip=\z@ \@plus 8em\pretolerance=10000 }}
\makeatother


%%%%
%% Custom theorem environments.
\newtheorem{mydef}{Definition}
\newtheorem{myexamp}{Example}

%%%%
%% Custom symbol definitions and abbreviations.
\makeatletter
\providecommand{\leadsfrom}{%
  \mathrel{\mathpalette\reflect@squig\relax}%
}
\newcommand{\reflect@squig}[2]{%
  \reflectbox{$\m@th#1\leadsto$}%
}
\makeatother

\renewcommand{\labelitemi}{$\circ$}
\newcommand{\edge}[1]{\stackrel{\bf{#1}}{\rightarrow}}
\newcommand{\ledge}[1]{\stackrel{\bf{#1}}{\leftarrow}}
\newcommand{\rel}[1]{\stackrel{\bf{#1}}{\leadsto}}
\newcommand{\trel}[1]{\stackrel{\bf{#1}}{\leadsto^*}}
\newcommand{\lrel}[1]{\stackrel{\bf{#1}}{\leadsfrom}}

\newcommand{\eqname}[1]{\tag*{#1}}% Tag equation with name

\newcommand{\nv}[0]{node(v)}
\newcommand{\ruleref}[1]{(\S\ref{#1})}
\newcommand{\apxref}[1]{(Appendix \ref{#1}.)}
\newcommand{\apxrefm}[3]{(Appendix \ref{#1}., \ref{#2}. és \ref{#3}.)}

%%% OWN PACKAGES

\usepackage{proof}
\usepackage{stmaryrd}
\usepackage{wasysym}
% \usepackage{mathtools}
\usepackage{minted}
% \usepackage{fontspec}
\usepackage{agda}

\usepackage{multicol}

\usepackage{tikz}
% \usepackage{forest}
\usepackage{float}

\usepackage{fontspec,unicode-math}
% \setmainfont{XITS}
% \setmonofont{XITS Mono}
% \setmathfont{XITS Math}
% \setsansfont{XITS}
% \setmainfont[
%  BoldFont={XITS Bold},
%  ItalicFont={XITS Italic},
%  BoldItalicFont={XITS Bold Italic}
% ]{XITS}

% \setsansfont[
%  BoldFont={XITS Bold}, 
%  ItalicFont={XITS Italic},
%  BoldItalicFont={XITS Bold Italic}
% ]{XITS}

\setsansfont[Scale=0.85,Path=fonts/,
BoldFont=DejaVu Sans Mono Bold Nerd Font Complete.ttf,
ItalicFont=DejaVu Sans Mono Oblique Nerd Font Complete.ttf,
BoldItalicFont=DejaVu Sans Mono Bold Oblique Nerd Font Complete.ttf
% Color={0019D4}
]{DejaVu Sans Mono Nerd Font Complete.ttf}

\setmonofont[Scale=0.85,Path=fonts/,
BoldFont=DejaVu Sans Mono Bold Nerd Font Complete Mono.ttf,
ItalicFont=DejaVu Sans Mono Oblique Nerd Font Complete Mono.ttf,
BoldItalicFont=DejaVu Sans Mono Bold Oblique Nerd Font Complete Mono.ttf
% Color={0019D4}
]{DejaVu Sans Mono Nerd Font Complete Mono.ttf}

% \usepackage{listings}
% \lstset{basicstyle=\ttfamily\footnotesize,breaklines=true,extendedchars=true}

\usepackage{newunicodechar}
% \newunicodechar{₀}{\textsubscript0}
% \newunicodechar{₁}{\textsubscript1}
% \newunicodechar{₂}{\textsubscript2}
% \newunicodechar{₃}{\textsubscript3}
% \newunicodechar{₄}{\textsubscript4}
% \newunicodechar{₅}{\textsubscript5}
% \newunicodechar{₆}{\textsubscript6}
% \newunicodechar{₇}{\textsubscript7}
% \newunicodechar{₈}{\textsubscript8}
% \newunicodechar{₉}{\textsubscript9}

% \newunicodechar{⌝}{\ensuremath{\neg}}
% \newunicodechar{△}{\ensuremath{\triangle}}
% \newunicodechar{△}{\ensuremath{\bigtriangleup}}
% \newunicodechar{▽}{\ensuremath{\nabla}}
% \newunicodechar{▽}{\ensuremath{\bigtriangledown}}
% \newunicodechar{}{\ensuremath{}}

% \newunicodechar{⊢}{\ensuremath{\vdash}}
\newunicodechar{⊩}{\ensuremath{\Vdash}}
\newunicodechar{⊪}{\ensuremath{\Vvdash}}
\newunicodechar{⊨}{\ensuremath{\vDash}}
\newunicodechar{⊫}{\ensuremath{\VDash}}

\newunicodechar{⇒}{\ensuremath{\Rightarrow}}
\newunicodechar{⇐}{\ensuremath{\Leftarrow}}
\newunicodechar{⇛}{\ensuremath{\Rrightarrow}}
\newunicodechar{⇚}{\ensuremath{\Lleftarrow}}

% \newunicodechar{▷}{\ensuremath{\rhd}}
\newunicodechar{↦}{\ensuremath{\mapsto}}
\newunicodechar{↣}{\ensuremath{\rightarrowtail}}
\newunicodechar{↪}{\ensuremath{\hookrightarrow}}
% \newunicodechar{}{\ensuremath{}}

\theoremstyle{definition}
\newtheorem{definition}{Definition}[section]

\newlength{\savedcolumnsep}

%%%%%%%%%%%%%%%%%%%%%%%%%%%%%%%%%%%%
%% Body section of Latex document %%
%%%%%%%%%%%%%%%%%%%%%%%%%%%%%%%%%%%%

\begin{document}
% \title{The title of your thesis}
% \thispagestyle{empty}
% \begin{center}
% {\Huge TDK dolgozat}\\[0.5cm]
% {\bf Név} \\[1cm]
% \end{center}


\begin{titlepage}
  \noindent
  \begin{minipage}{0.25 \textwidth}
    \includegraphics[height=40mm]{figures/cimer.png}
  \end{minipage}
  \hfill
  \begin{minipage}{0.67 \textwidth}
    \large
    Eötvös Loránd University \\
    Faculty of Informatics \\
    Department of Programming Languages and Compilers \\
    
  \end{minipage}

  \vfill

  \begin{center}
    {\LARGE \bfseries Formalizing a relational model of concurrent programs in a dependently typed environment}
    \\[1.5cm]
    {\Large TDK dolgozat / Diplomamunka}
    \\[3cm]
    % \\[6cm]
    \begin{minipage}[t]{0.45 \textwidth}
      \emph{Supervisor:} \\[0.25 \baselineskip]
      {\large Ambrus Kaposi} \\[0.5 \baselineskip]
      Assistant professor
      \vspace{1cm}
      
      \emph{Supervisor:} \\[0.25 \baselineskip]
      {\large Melinda Tóth} \\[0.5 \baselineskip]
      Associate professor
    \end{minipage}
    \begin{minipage}[t]{0.45 \textwidth}
      \begin{flushright}
        \emph{Author:} \\[0.25 \baselineskip]
        {\large István Donkó} \\[0.5 \baselineskip]
        Computer Science MSc \\ %% The name of your program
        2. year
      \end{flushright}
    \end{minipage}
  \end{center}

  \vfill

  \begin{center}
    \large Budapest, 2020
  \end{center}
\end{titlepage}

%%%
% Questions:
% + Hogyan kell korrektül megfogalmazni, hogy az is nagy szerepet játszott az Agda választásában, hogy Ambrus abban tud a legtöbbet segíteni?
% + Mathematical foundation: determinisztikusan csináljuk, ugye? (HZ van, ahol ezt, van ahol azt használja, lásd: puma.pdf logikai függvény vs jegyzet) Plusz kellenek ezek a végtelen állapotlisták? --> Footnote, the model is more general and uses relations, we work with functions.
% A chapter-eknél is elvárás, hogy egy cím alatt ne következzen egyből másik cím?
% + Próbáltam a subject részt direkt máshogy leírni, mint HZ cikkeiben van, de ettől csak sokkal kavarosabb lett...
% + TODO: Why invP instead of P \in inv?
% + Amúgy miért kell ennyire általánosan? Miért nem lehet |F(b)| = 1 mindig (tehát függvény)?
% + A background-ban a subject-hez azokat a részket is leírjam, amik még nincsenek implementálva?
% + Figures / Numbers? But what if they are separated line by line?
% + Full source code at the end in appendix? What should go inline?
% + "Ne a kód legyen kommentelve, hanem a magyarázat illusztrálva" arány?
% + PSP actual meaning, FIGURE!
% + Amúgy lehet csatornák csatornájáról beszélni? Plusz mi történik lov([]) és lorem([]) esetén?
%%%

%%%%%%%%%%%%%%%%%%%%%%%%%%%%%%%%%
%% Content sections start here %%

\begin{abstract}
%% While you can write all your content here in the main file, it's recommended
%%   to keep your content into separate files. The \input directive simply
%%   copies here the text of the pointed line.
Sequential programming languages have already been formalised in dependently typed programming languages, such as for example Agda or Coq, but the formalisation of concurrent programs is still in its early days. The goal of our research is to formalise a relational model that describes the behaviour of distributed concurrent programs in a computer based theorem prover system. Our long term goals include the verification of the material of the subject titled \textit{"Specification and Implementation of Distributed Systems"}, which serves as a core part in the Computer Science education at Eötvös Loránd University in Budapest.

\end{abstract}

\tableofcontents

\chapter{Introduction}
\label{chp:introduction}

Software plays a critically important role in the life of modern societies. More or less everybody interacts with computer programs on countless occasions during our everyday lives, most of the time probably not even noticing. For example, just paying with a credit card while shopping, being able to call someone with our mobile phones, or even just as mundane tasks, as operating modern versions of basic home appliances, like washing machines or microwave ovens requires interaction with software.

% https://youtu.be/ecIWPzGEbFc?t=4253

During the COVID-19 pandemic in the spring of 2020 we also became more aware of how much we are relying on software for online communication to be able to do our work remotely and keep in touch with friends and family. Many of us have experienced during these times how frustrating it can be, when programs do not work as intended.

(Another clear indication of the importance and reliance on computers is the crash of the United States' unemployment system, which could not handle the huge influx of applicants due to the quarantine.)

Most of these examples are just about convenience factors, but if we take the amount of more critical scenarios into account, such as for example software running on the computer system of an airplane, or even keeping a nuclear power plant safe, we can see that programmers have an even bigger impact. (Self driving cars could also be mentioned, but machine learning is a whole another field with its own set of ethical and moral questions.) The economy of the world depends on software, the rulers of nations rely on computer systems at their disposals. To quote from a 2016 talk given by famous American software engineer, Robert C. Martin (one of the authors of the Agile Manifesto), titled "The Future of Programming", he claimed: \textit{"Civilization depends on us."} and even went as far as to say that:

% 1:14:27
\textit{``We rule the world. The world doesn't know this yet. We don't quite know it yet. Other people believe that they rule the world, but they write the rules down and they hand them to us, and then we write the rules that go into the machines that execute everything that happens on this planet nowadays. No law can be enacted without software, no law can be enforced without software. No government can act without software, we rule the world.''}

I wouldn't make such an extreme claim, that developers rule the world, but I certainly agree that they play a significant role that comes with high responsibilities.

\section{Motivation}

Borrowing from the same talk, we can define the beginning of programming around the work of Alan Turing, since he was the first one, who wrote code for machines in the sense that we would recognize today. His work has undoubtedly played a crucial role in laying down the foundations of programming. Turing machines are a core part of computer science and are still taught in university courses today. In his time, he described the future possibilities of his vision with the following sentences in a lecture to the London Mathematical Society\cite{turing-lecture}.
% https://youtu.be/ecIWPzGEbFc?t=1350

\textit{``In order to supply the machine with these problems we shall need a great number of mathematicians of ability. These mathematicians will be needed in order to do the preliminary research on the problems, putting them into a form for computation.''}

He stated the need for mathematicians for the precise formalization of problems. This need has since been abandoned and most of software products nowadays are just developed through trial and error processes, being patched until they pass all defined test cases, but nobody can really be assured of their correct behavior. For some types of applications, such as for example games this can be acceptable, because the worst outcome of bugs are frustrated end users, but for more critical systems, the correctness of which can decide between life and death, that is simply not enough. We need to have formal strategies to verify behaviors of programs under all circumstances instead of just observing them for the most likely situations.

There are lots of existing means for confirming the adherence of simple sequential programs to their specifications, ranging from formal verification procedures carried out on paper to contracts built into programming languages, that can be checked and enforced automatically, either via static code analysis, or during runtime by monitoring different values. Reasoning about parallel programs is a lot more complicated, but this complication also serves as an explanation for the need to do so, since concurrency is often a result of multiple systems working together, in which case it is a lot easier to make mistakes because of the unpredictable order of execution of instructions. Several different ways are known to approach formal proofs of correctness for concurrent programs. For example multiple specific methods can be seen in \cite{hons_1202}. What we chose to base our research on is the material of the subject titled \textit{"Specification and Implementation of Distributed Systems"} which is discussed in more detail in Section \ref{sec:subject}.
%builds upon Hoare Logic \cite{hoare-logic} and the work of \cite{Lamport1980}.

Our formalization does not follow the material exactly, but we tried to stay as close to the original notation as possible, as we also have intentions to later further expand this project to cater for educational usage for example as part of the practical courses. We consider this to be a valuable opportunity for creating a teaching tool that can greatly aid the understanding of the subject for students. The use of proof assistants have already been successfully introduced in several other classes\cite{formalsemantics-typesystems}, which helps making this idea seem quite feasible.

\section{Outline}

In chapter \ref{chp:related} we present several related works, with the most similarities and differences of the most akin article discussed in more detail. After that in chapter \ref{chp:background} the necessary background knowledge for the subject and the tool of our formalization is introduced. Following this, in chapter \ref{chp:method} our approach to the problem and some experiences encountered through our project are described. Our results are presented in chapter \ref{chp:results}, separated into the formalization which laid down the foundation and the proofs that were built on top of it. Finally in chapter \ref{chp:conclusions} we conclude our research until now and discuss the possible ways in which it could potentially be extended in the future.

\section{Results}

The main results of our work include the fully formalized version of a big core part from the original model we built our research around, which is precise enough for computer based typechecking. On top of this foundation we have developed proofs for several generic lemmas and theorems. Also included in the article is the formalization of a parallelized version of the bubble sort algorithm and the verification of some of its specification properties.

% \input{examples.tex}

\chapter{Related Work}
\label{sec:related}

Since our goal is not the introduction of a new way to address the problem of creating correctness proofs for parallel programs, rather the adaptation of an existing system for formalized implementation, instead of discussing the theoretical methods, here we focus on a more practical approach and explore what others have achieved in the field of computer based verification.

\section{Previous works in Agda}
A very recent article \cite{Bergsten2017MethodsFU} presents a similar approach to ours, by setting out to provide an alternative to paper based proofs by formalizing an existing semi-formal model of concurrent programming. Their results provide methods for the confirmation of certain safety and liveness properties. Since it was the most closely related to what we have done, we compare their work to ours and discuss the similarities as well as the differences. The Agda programming language served as the meta-theory, in which the formalization and proofs were constructed for both projects, but the object-theory was different. We built our system around a language similar to UNITY, a language specifically created for and entirely based on parallelism for a book titled \textit{Parallel Program Design: A Foundation}\cite{misra1989foundation} (not to be confused with the modern game engine), that has essentially no guaranteed sequentiality, while their subject was an extension of the CPL language borrowed from \cite{owicki1982proving}, upon which they based most of their work, which is to some extent a regular imperative sequential language that has added constructs which introduce the capability for concurrency. This difference is also made very visible by their \textit{Proof by Control Flow} method, which implies that there is a distinct order present in the execution. Another big difference is the main focus of the projects. The aim for creating safety proofs was shared, but their other target, liveness proofs were not among our top priorities (although the fixed point property among our results can be related to it), rather we concentrated more on the formalization of general theorems as well as confirming adherence to specification, which is not included in their work. Further differences can be found by observing that the methods they utilize are in closer relation to operational semantics, while our formalization describes behavior in a manner more similar to denotational semantics. Also, some of the proofs presented in that paper are not completely constructive, some properties are only checked by functions essentially returning Boolean results, thus no so-called witness can be used for explaining or reasoning about the results, while the results that are presented in this current paper aim to all be constructive. They also did not implement every little detail included in their proofs, some steps include the usage of postulated properties, which are either relying on paper-based proofs or intuition. Our system is fully self-contained and is built from the ground up in a way meant not to include any external dependency. Their work was related to and part of it was further expanded in \cite{Johan2018ProofCF}.

\section{Previous works in Isabelle/HOL}
Another paper\cite{Complx-Isabelle} discusses a similar topic, but uses the Isabelle/HOL proof assistant\cite{nipkow2002isabelle} built on the imperative paradigm instead of the type theory based alternatives. Their main subject is a version of the generic SIMPL programming language originally introduced in \cite{simpl-schirmer2006verification}, that they extend in a way to support parallel composition and synchronization through shared-variable concurrency and called it COMPLEX. This still results in a system, which is sequential in its nature and contains concurrent sections. This approach is fitting for their goals, as one of their future aims is the translation of low level C code into COMPLEX for verification of for example concurrent operating system kernels. Their work is inspired by the Hoare-Parallel framework\cite{hoare-parallel-nieto2002verification} which is a formalization built using the Owicki-Gries method\cite{owicki1976axiomatic} in Isabelle/HOL. The Owicki-Gries method extends the verification procedure of sequential programs by introducing the notion of \textit{interference freedom}. It works by first proving each single thread correct on its own and then the atomic instructions of the threads are proven not to interfere with the correctness of each other.

\section{Previous works in Coq}
A case study\cite{coq-concurrent-verification-case-study} shows that there are many advantages to the library based approach, which eliminates some complicated modeling tasks by implementing a generalized version of parallelism instead of aiming straight for the proof of a concrete program. They investigate these properties through the verification procedure of a mail server originally written in Java by rewriting it in a modeling language provided by a library. \cite{coq-mail-server} The development of the Coq library is discussed in \cite{AFFELDT200817}. It contains a modeling language based on $\pi$-calculus
% \cite{pi-calculus-Milner1992ACO}
\cite{MILNER19921}
\cite{MILNER199241}
\cite{sangiorgi2003pi}
(which has been modified to enable the usage of Coq datatypes and control structures), a specification language and a collection of reusable lemmas. In this sense, it is similar to our approach, as these mentioned components are also present among our results. Models formalized in their library are also suitable for extraction into executable OCaml code using the facilities provided by Coq, although it is not fully automatic and requires manual modifications.

\section{Our work}
Our approach is different from these examples in that its main goal is to aid the development of solutions given through the formalized specification language, not necessarily the verification of already existing codebases. The control flow (or more like the lack thereof) is also a big differentiation factor of our model when comparing to other similar systems.

\section{Proof assistant for parallel systems}
In another article\cite{Mauw1991APA}, that is more on the theoretical side, but we still considered relevant, the description of a proof assistant based on the algebra of communicating processes, specifically designed for proofs around parallel systems is given.



% In \cite{proving-parallel-assertions} we can see the application of such methods to a concrete program, several properties of an elementary airline reservation system implemented in a general parallel programming language are verified.

% TODO: min. 2 oldal

\label{chp:related}

\chapter{Background}
\label{chp:background}

In the following chapter we introduce the material that we based our formalization on and the system we chose as our meta-theory. We do not cover the whole model in detail, but mostly focus on the concepts that are needed for understanding the results of our work. A short introduction to Agda, the language of our choice can also be found at the end of the chapter.

\section{Subject: Specification and Implementation of Distributed Systems}
\label{sec:subject}

The subject of our formalisation is a part of the material from the \textit{Specification and Implementation of Distributed System} course taught at ELTE. The main goal was following the official notes of the subject\cite{hz-orsi} as closely as possible, but since the code needed to be more precise than written definitions, some constructs were altered. The system extends a relational model defined in \cite{fothi-prog} which itself is built upon Hoare-logic \cite{hoare1978proof}\cite{hoare-logic}, more specifically concurrent Hoare-logic\cite{Lamport1980}. Proofs for sequential programs are handled using pre- and postconditions with the devices provided by first order logic. This is expanded to parallel execution in a similar fashion to UNITY. Abstract programs are formalized through a set of conditional assignments the execution of which are represented as a list of state transitions that can be considered as trees, which results in a branching-time temporal logic \cite{emerson1988branching} (more specifically computation tree logic) like system.

An abstract program as defined by this system can be thought of as an initialization and a set of further conditional instructions. These are then selected randomly and if their condition is satisfied in the current program state, they are executed. This can be done in any arbitrary order, their behaviour stays the same if the scheduling is impartial.

After laying down the foundation for constructing programs and reasoning about their behavior, the subject introduces a formal concept of task specification and then defines the correctness of a program by proving that it adheres to those constraints posed by the specification. We iterate on these constructions and implement them, so the verification of the adherence can be done on the computer, thus avoiding potential human errors that can occur while carrying out such a process on paper.

There are multiple ways, in which said correctness of a program can be shown. Simply giving a proof, that all the imposed criteria are held is always sufficient, but gets inconvenient for larger programs. To avoid the complication of proofs, one can utilize the method of reduction to an already known problem. If there is a recurring pattern, general statements about its behavior can be proven, so that these do not need to be done each and every time the pattern is used. Just showing how the currently examined part of the program corresponds to the pattern is enough instead. Another technique for avoiding repetition and introducing segmentation into proofs to keep their brevity is to construct programs from other smaller programs, by for example taking their union, essentially using them as so called \textit{building blocks}.

\subsubsection{Mathematical foundation}

The relational model
finite, infinite lists of states
state transition function
logical function = condition


\begin{definition}{Interleaving semantics}
\label{def:interleaving-semantics}
In a model that conforms to interleaving semantics, for every valid parallel execution, there also exists a linear execution path of the same instructions, that achieve equivalent results.
\end{definition}

\begin{definition}{Impartial Scheduling}
\label{def:impartial-scheduling}
A scheduling is called impartial if it ensures that running the program indefinitely will result in the selection of each and every conditional instruction infinitely many times.
\end{definition}

The execution units can be thought of as processing cores, or even several computers working together in a cluster.



\section{Type Theory}
In the following section we will discuss what Type Theory is, how it enables formalization of models with proofs, to be checked by computers and explain our choice of the Agda implementation.

\subsection{General Principles Overview}

Type theory is an alternative foundation for mathematics, which enables the formalization of constructive proofs through the connections to intuitionistic logic given by the Brouwer–Heyting–Kolmogorov interpretation. After formalizing a model by defining its types and their elements, one can express statements and theorems in forms of new types, the instances of which can be thought as of proofs for them.
This is due to the so called \textit{"propositions-as-types"} paradigm, formally known as the Curry–Howard isomorphism.

As an example, given two statements, $A$ and $B$, the function with the type signature $A \to B$ represents the theorem which claims that every proof of $A$ can be mapped to a proof for $B$, thus $A$ ensures $B$. If an implementation for a function with said type is given, the theorem can be considered the be proven, since if we are in possession of a proof for $A$, we can execute it and obtain a proof for $B$. The fact that it does not only claim, that such an instance exists, but gives one right away instead is called a constructive proof.

\subsection{Implementations}
There are several existing programming languages that implement type theories, thus are capable of describing theorems and constructing proofs for them. They achieve this using dependent typesystems, in which types can not only be parametrized over other types, but also values. Out of these we tried out the three most popular alternatives.

\subsubsection{Idris}
Idris\cite{Brady2013IdrisAG} is a relatively new contender in the field of programming with dependent types. Its development is led by Edvin Brady with the aim of creating a general purpose language.

We first tried implementing our formalisation in Idris, as it seemed to be the freshest language. It also has good support for every major operating system, wider editor compatibility than the alternatives and in some sense a more modern standard library, which is easier to discover due to the in our opinion better documentation.

Under certain circumstances the implicit parameter handling of Idris seemed to better for us than that of Agda, and there are constructs, such as for example heterogeneous vectors that are easier to implement in Idris, because it uses cumulative universes, but
after reaching a certain complexity, the type checker of Idris unfortunately turned out to be too slow, so we moved on to other languages. It also lacks Unicode support, with perfectly valid and understandable reasons outlined by Edvin Brady, but it also makes the code a lot more verbose and harder to integrate with the original notation.

% ?Idris also has heterogeneous vectors which are harder to create in Agda due to its implementation of universe polymorphism.?


\subsubsection{Agda}
Agda is a dependently typed programming language that was originally described in the PhD thesis of Ulf Norell\cite{norell:thesis} and later completely rewritten for a second version. It mainly follows the style of Haskell.

Its conventions of highly relying on Unicode symbols for identifiers helped in staying similar to the original notation of the curriculum of the subject.

\subsubsection{Coq}
The Coq proof assistant\cite{} is the earliest of the three, even predating Agda by ten years. (The initial version was released on the 1st of May, 1989.) It is really well established and focuses mainly on creating proofs using a tactic language instead of using regular functional programming constructs.

\chapter{Method}
\label{chp:method}

The development process was done in an iterative style, meaning that we had to backtrack multiple times and rewrite certain parts over and over, but this often helped in finding better solutions. We believe that experimenting with multiple variations instead of accepting the first one that works is vital for converging to a better result instead of achieving a certain goal earlier only to move on. This process can lead to code that is more expressive, concise, easier to understand and thus maintain and explain, reason with and about.

This approach also results in the later layers built on top of earlier parts having a better foundation. The complexity of the definitions of data types and the functions describing their behaviors directly impacts the complexity of the proofs written for statements about them. Finding a simpler, but equivalent way to formalize a definition, that still results in equivalent semantics can lead to greatly simpler proofs.

After formalizing certain constructs and writing some proofs in connection with them it is usually much easier to see what causes complications and how to modify them to make their handling easier. These discoveries have sometimes led to decisions that diverged from the original material, warranted by the differences in required precision between paper based proofs, where certain parts can be omitted and computer based proofs, that need every little detail to be verified.

Sometimes the formalization of certain parts led to deeper understanding of their behavior, because of which we were able to restructure the definitions and proofs in a way that helped incorporate more and more elements from the Agda Standard Library\cite{agda-stdlib}, thus reducing the amount of added code and representing parts of our formalization as specialized versions of more generic concepts.

The work was done in smaller segments, which made it feasible to build and extend the model step-by-step instead of having to write one huge monolithic codebase all at once without being able to test smaller parts of it. This approach sometimes had drawbacks, when adding a new element required deep modifications in the existing parts, but overall it helped in advancing the project at a steady pace.

We started with an implementation of a highly simplified model in Idris, that served as an initial proof-of-concept. The main outlines were laid down and valuable experience was gained from discovering different possible workflows for the transformation of paper based mathematical definitions into code. Some simple proofs were written using that, but after trying to expand to a more complex model, the type checking of Idris seemed to be too slow, which led to the decision of switching to Agda.

Luckily the similarities of Agda and Idris meant, that our previous work was not lost, because it was easily translatable into the new environment. Parallel to the development of the base model we started including some statements in our code as well, which we prove for the current state of the codebase, and seeing if they were still true after modifications to the underlying model helped us determine if the introduction of any given change would break the consistency of the model or not.

% With helyett segédfüggvény -> könnyebb bizonyítani
After proofs started getting longer and more complicated, we started looking into ways to simplify and shorten them. For example we were able to achieve cleaner types during the interactive hole based proof process after eliminating dependent branching from them by rewriting some semantic functions to use auxiliary functions instead of using the built-in with-abstraction provided by Agda.

% Listák függvényként ábrázolása -> kiértékeli + rövidebb
We managed to further simplify proofs by moving the representation of arrays from lists to functions, which map from the set of natural numbers to the type of the contents of the array. By introducing this Agda could rely on the fact that functions always evaluate and simplify the proofs by computing the required types automatically, instead of us having to do pattern matches on the constructors of lists, thus several proofs became much shorter.

% \bN helyett Fin ("Az állapottér véges sok legfeljebb megszámlálhatóan végtelen típusértékhalmaz direktszorzata [Fót 83].")
The iterative method also applied to our aim of increasing the precision of the formalization in terms of similarity to the original material where it was possible. Sometimes it was easier to experiment with simplified versions first, which then could be later adapted in their details to be more consistent with the subject. An example for such is the moving of variable indexes to finite sets instead of the initially introduced state space that was mapping from natural numbers. This was done as a step towards conformance with the statement defining state space as a Cartesian product of finitely many type value sets. We generally tried to stay as close as possible, but we encountered some parts, which required modifications, because they were designed to be convenient for human reasoning, but are not formal enough to be sufficient for type checking.

% Assertionben implementált külön-külön \lfloor és \rfloor helyett Decision bevezetése -> minden sokkal könyebb / rövidebb
Another philosophy, that we tried to follow was the reduction of duplication by extracting and reusing code that appeared in multiple places. One example for this was the handling of Predicates, Conditions and Assertions. Originally, we had the \textbf{Predicate} inductive datatype, which described certain constraints that could either be checked if they evaluate to true in a certain state by converting them to a \textbf{Condition}, or a proof could be given that claims that they are satisfied by converting them to an \textbf{Assertion}. (More details about this will be covered in Chapter \ref{chp:results}.) These conversions had common parts which we later extracted into an intermediary representation by introducing the \textbf{Decision} type, which unified the shared part of the semantics and could later be turned into either a \textbf{Condition} or an \textbf{Assertion} more easily.

\newpage

A similar change was introduced, when we split the definition of \textbf{Ensures} ($\mapsto$) by introducing \textbf{Progress} ($\rightarrowtail$), which allowed us to construct partial proofs that applied to \textbf{Progress} on its own and later utilize them as parts of the proofs for \textbf{Ensures}. This resulted in a better structured and more readable code.

When introducing a new construct, it was often implemented in a very specific way at first, which aided the initial implementation, but after gaining a better understanding of the behavior and meaning of that certain part, by for example writing some proofs for it, they were often rewritten in a more generic fashion in order to make the model as extensible as possible, for example only introducing arrays of natural numbers first and later extending the array type to be parametrized with the specific set of types defined.

As another example for this, adding general equality of expressions as a predicate instead of only comparing natural numbers required the introduction of the notion of our own equality, since decidable equality for functions cannot be implemented, and that was needed due to the conversion of predicates to conditions. (This would have been necessary anyway when introducing data channels, since their equality is defined as a quotient with regards to their actual contents, their history does not matter during comparisons.) This generic decidable equality made the storage of array lengths necessary as well, since the comparison function needed to know which part of the function's domain it has to evaluate. When we replaced the built in equality with our own alternative, we used a function at first, that mapped to a different type for each datatype constructor (for example it simply mapped back to the built-in equality for natural numbers), but that turned many implicit variables that were previously automatically solved in our proofs unsolvable by Agda, thus we changed the representation to an inductive datatype with a single constructor instead of the function and successfully aided Agda in solving the metavariables this way.

% ownEq -> ownDecEq -> Arrays need length
% ownEq as datatype -> Statements work again without implicits

If we had tried to do all of these changes at once, it would have been easy to get lost dealing with multiple design decisions at the same time and not really seeing a clear way forward, but dealing with them step by step allowed us to keep the tasks in a visible scope and not lose track of our goals.

% s0 implicit eltűnik, mert nincs rá szükség -> nem tudja kitalálni az Agda
After formalizing statements that describe some safety and progress properties of programs we noticed that there was a certain implicit variable that Agda had trouble figuring out on its own in multiple scenarios. This was the $s_0$ initialization instruction, the first part of the representation of a parallel program, which the original material defines as a pair made out of a single initialization instruction and a set of conditional instructions that can be executed afterwards in any order. By observing the definitions we realized that it is indeed essentially irrelevant at the later stages of the program and only plays a role in constructions dealing with the beginning, such as \textbf{INIT} and \textbf{inv}. By splitting up the definition into the separate notions of \textbf{ParallelProgram} and \textbf{InitializedParallelProgram}, we were able to avoid the extraneous inclusion in the properties for which it was unnecessary and eliminate the ambiguous implicit variables.

As the credibility of the results that are proven in such a system depend greatly on the soundness of the underlying implementation, we have also added some safeguards in forms of assertions that describe the desired behavior of the implemented semantic functions. While this does not fully cover all the possible mistakes, by constructing instances for these types, we confirmed that some important properties hold for the foundation upon which our further work is based.

Finally, we have prepared the system for the inclusion of data channels, by implementing a type called \textbf{QueueWithHistory} and the functions that are necessary for handling the manipulation of its representation. The particularity of channels in the sense, that they act more like objects with methods makes the description of their behavior with just simple assignments complicated, which is undesirable, since it would greatly increase the complexity of not only their definitions which in turn also makes it more prone to mistakes, but also undermine the clarity of proofs based around them. Because of this, we had to change the representation of instructions from a condition paired with a list of variables with value expressions to be assigned. Instead we now have a condition paired with a list of instructions, that will include the extension (enqueue) and removal (dequeue) operations for channels. The problem is that channels would need to be excluded from assignments, so their values cannot be arbitrarily overwritten. Due to time constraints we were not able to devise a clean method that achieves the correct behavior, thus proofs about channels are outside of the scope of this current paper.

% TODO: Mennyire szabad kritizálni az ORSI anyagot?
% TODO: Amúgy lehet csatornák csatornájáról beszélni? :thinking:
% We encountered several problems when we started introducing data channels, which are a fundamental type of variable in the original material.
% DataChannel eq needs quotient, history does not matter -> ownEq


\chapter{Results}

\section{Formalization}

Since our formalization does not follow the book of the subject exactly, we present the five basic notions we introduced in our version:
\begin{itemize}
    \item Type - Inductive set of possible types of variables in the program.
    \item Variable - Index that can be used to reference parts of the state.
    \item State - A function that maps from the variables to the value sets of their types.
    \item Predicate - Inductively defined assertions.
    \item Instruction - A list of parallelly executed assignments that give certain variables new values by evaluating expressions in the current state.
\end{itemize}

% Semantic brackets macro
\newcommand{\sem}[2]{\llbracket #1 \rrbracket #2}

There are also a few extra semantic operators that give the constructs meaning:
\begin{itemize}
    \item $\sem{\_}{c}$ - Turns predicates into conditions that can be evaluated for a given state. Essentially turns a predicated into a function from states to boolean logic. ($\{true , false\}$) \\
    ($\sem{\_}{c} : Predicate \to State \to Condition$) where $Condition = State \to Bool$

    \item $\sem{\_}{a}$ - Makes a statement from a predicate by turning it into a function from state to type. ($\{\top , \bot\}$) By creating a type, an instance can later be given, proving that the statement and thus the predicate holds in the given state. \\
    ($\sem{\_}{a} : Predicate \to State \to Set$)

    \item $\sem{\_}{i}$ - We also define an additional operator which turns instructions into functions that map from states to states. \\ ($\sem{\_}{i} : Instruction \to State \to State$)
\end{itemize}

The following constructions can be introduced based on them:
\begin{itemize}
    \item Truth set - The truth set of a condition is the set of states in which the logical function evaluates to true ($\lceil P \rceil = \{st : State | P(st) = true\}$).
    \item Implication - One condition implies another ($P \Rightarrow Q$) if its truth set is subset of the others ($\lceil P \rceil \subset \lceil Q \rceil$).
    \item Weakest precondition - The weakest precondition for a given condition with respect to an instruction is the logical function that only evaluates to true for a certain state, if after executing that instruction the condition hold true. $wp(i , P) = P \circ \sem{i}{i})$.
    \item Strongest postcondition - For a given condition and instruction the strongest postcondition gives the logical function that determines after executing the instruction whether it was started from a state that satisfied the original condition.
    $
    sp(i, Q)(st) =
        \left\{
        	\begin{array}{ll}
        		true & \mbox{if } \exists st0 : State : \sem{Q}{st0} = true \land \sem{i}{i}(st0) = st \\
        		false & \mbox{otherwise}
        	\end{array}
        \right.
    $
    
\end{itemize}

To iterate on these, we can add some conditional versions of them:

\begin{itemize}
    \item Conditional Instruction - A conditional instruction is an instruction paired with a condition under which it can be executed. ($(P, i)$)
    \item Conditional Instruction Effect Function - 
    $
    \sem{(P, i)}{ci}(st) = 
    \left\{
        	\begin{array}{ll}
        		\sem{i}{i}(st) & \mbox{if } \sem{P}{c}(st) = true \\
        		st & \mbox{otherwise}
        	\end{array}
        \right.
    $
    
    \item Conditional Weakest Precondition - $cwp(ci , P) = P \circ \sem{ci}{ci}$.
    \item Conditional Strongest Postcondition - \\
    $
    csp(ci, Q)(st) =
        \left\{
        	\begin{array}{ll}
        		true & \mbox{if } \exists \, st0 : State , \sem{Q}{c}(st0) = true \land \sem{ci}{ci}(st0) = st \\
        		false & \mbox{otherwise}
        	\end{array}
        \right.
    $
\end{itemize}

And finally the parallel constructions:

\begin{itemize}
    \item Parallel Program - A pair of an initial conditional instruction ($s_0$) and a set of conditional instructions. ($(s_0, \{s_1, \, s_2, \, \dots , \, s_n\})$)

    \item Parallel Conditional Weakest Precondition - Logical function that only returns true, if all the conditional instructions of a parallel program satisfy the Conditional Weakest Precondition. \\
    $
    pcwp(s , P)(st) = \bigwedge\limits_{s_i \in S} cwp(s_i, P)(st)
    $
    \item Parallel Conditional Strongest Postcondition - \\
    $
    pcsp(s , Q)(st) =
        \left\{
        	\begin{array}{ll}
        		true & \mbox{if } \forall s_i \in s : csp(s_i, Q)(st) \\
        		false & \mbox{otherwise}
        	\end{array}
        \right.
    $
\end{itemize}

Using these functions, the following statements can be created:
\paragraph{Unless:}
$P \rhd_S Q$ means that the condition P either stays satisfied, or Q becomes true.
% $P \land \neg Q \Rightarrow \mbox{PCWP}(S, P \lor Q)$
% $P \land \neg Q \Rightarrow \mbox{pcwp}(S, P \lor Q)$

\begin{equation}
    \label{eq:unless}
    \infer{P \rhd_S Q}{P \land \neg Q \Rightarrow pcwp(S, P \lor Q)}
\end{equation}

\paragraph{Progress:}
$P \rightarrowtail_S Q$ means that there is at least one conditional instruction in the program through which we can move from a state which satisfies $P$ but not $Q$ to a state that satisfies $Q$.

\begin{equation}
    \label{eq:progress}
    \infer{P \rightarrowtail_S Q}{\exists s_i \in S : P \land \neg Q \Rightarrow pcwp(s_i, Q)}
\end{equation}

\paragraph{Ensures:}
$P \mapsto_S Q$ means that we can only leave the truth set of $P$ through $Q$ and there exists a conditional instruction through which this can actually happen. (And due to the impartial scheduling, see Definition \ref{def:impartial-scheduling}, it will eventually be executed, so being in the truth set of $P$ essentially ensures that we will step into the truth set of $Q$ in the future.)

\begin{equation}
    \label{eq:ensures}
    \infer{P \mapsto_S Q}{P \rhd_S Q & P \rightarrowtail_S Q}
\end{equation}

\paragraph{Inevitable:}
$P \hookrightarrow_S Q$ is the transitive disjunctive closure of $P \mapsto_S Q$, meaning that if we ever step into the truth set of $P$, we will inevitably have to reach a state in the truth set of $Q$, even if through numerous intermediate steps.

\begin{equation}
    \label{eq:leadsto-fromensures}
    \infer{P \hookrightarrow_S Q}{P \mapsto_S Q}
\end{equation}
\begin{equation}
    \label{eq:leadsto-transitivity}
    \infer{P \hookrightarrow_S R}{P \hookrightarrow_S Q & Q \hookrightarrow_S R}
\end{equation}
\begin{equation}
    \label{eq:leadsto-disjunctivity}
    \infer{P \lor Q \hookrightarrow_S R}{P \hookrightarrow_S R & Q \hookrightarrow_S R}
\end{equation}

\paragraph{Invariant:}
A $P$ predicate is invariant in an $S$ program with the $Q$ initial predicate if $Q$ implies $P$, so it must be true at the beginning, and being in the truth set of $P$ guarantees that we will forever stay there. \\
$Invariant_S(P, Q) = (csp(S, Q) \Rrightarrow \sem{P}{a} \, \land \, P \Rightarrow pcwp(S, P))$

\paragraph{Fixed point:}
We call a certain state ($st$) a fixpoint of a program, if no conditional instruction can further change the values of variables, which means that execution can be halted. \\
$st \in fixpoint(S) = \forall s_i \in S, (\sem{s_i}{ci}(st) = st)$ (also denoted: $\varphi(S)$)


\subsection{Compositions}

\subsubsection{Union}

\subsection{Programs}
\subsection{Statements}
\subsection{Constructions}

\section{Proofs}

There were two kinds of statements that that we could prove in our system. First, there were general theorems, which are true independently and then there is the correctness of certain specific programs, by which we mean the adherence to their specification.

\subsection{Theorems}

\subsubsection{Reflexivity of $\hookrightarrow$}

\subsubsection{PSP}

The most complicated theorem we prove was the theorem of "PSP" (see equation \ref{eq:psp}), meaning "Progress-Safety-Progress" (introduced in this form in \cite{Chandy1988ParallelPD}), which states that if from a set of starting points $P$ a program must eventually reach a set of result states $Q$ and the truth set of a given predicate $R$ cannot be left without entering $B$ than in case we begin execution from a shared point of $P$ and $R$ we will either get to a result ($Q$) while staying in $R$ or enter an error state, $B$. We prove several other helpful lemmas first, that facilitated the construction of the bigger proof later.

\begin{equation}
    \label{eq:psp}
    \infer{P \land R \hookrightarrow_S (Q \land R) \lor B}{P \hookrightarrow_S Q & R \rhd_S B}
\end{equation}

For the construction of this proof we used structural induction. We divided the goal into three parts based on how $P \hookrightarrow_S Q$ was constructed (from an Ensures statement, by the rule of transitivity or distributivity). The first goal is then solved by proving that equation \ref{eq:psp-ensures} holds and then constructing the goal through that.

\begin{equation}
    \label{eq:psp-ensures}
    \infer{P \land R \mapsto_S (Q \land R) \lor B}{P \mapsto_S Q & R \rhd_S B}
\end{equation}

The second and third subgoals can be solved by recursively applying the theorem.

\subsection{Correctness}

We implemented a bubble sort algorithm in the language, but were so far only able to prove a few properties of it, not the full correctness yet.

% We set out to implement bubble sort, which we achieved, but until now we only
We managed to prove that if the first element of the array is one, it can later only change to zero.

\begin{equation}
    \label{eq:bubble-proof-1}
    a[0] = 1 \rhd_{BubbleSort_n} a[0] = 0
\end{equation}
Where BubbleSort is defined as it can be seen in \cite{hz-orsi} and \cite{hz-article}.
% \begin{equation}
% \begin{multlined}
\begin{align}
\begin{split}
    BubbleSort_n = ( & \\
      & (TRUE \:,\: SKIP), \\
      & \{ \, ( \, (a[i] > a[i+1])\:,\:(a[i],a[i+1]:= a[i+1],a[i] \, ) \; | \; i \in \{1,\dots,n-1\} \, \} \\
    ) &
\end{split}
\end{align}
% \end{multlined}
% \end{equation}

\chapter{Conclusions}
\label{chp:conclusions}

Our formalization turned out to be adequate for proving theorems about the system, and while building proofs of correctness for more complicated programs is still a very tedious process, we have already managed to simplify it in several ways, which allowed us to verify partial properties. There might be even more opportunities available for making the implementation easier to work with that we have not noticed yet.

Find our source code at the \verb|tdk| branch of our GitHub repository: \url{https://github.com/Isti115/orsi-formalization/tree/tdk}.

\section{Further Work}

So far we feel like a good foundation has been laid down, but it is not nearly complete and has many ways, in which it can and should be continued. We discuss some of these in this section.

\subsection{Expanding the model}

The original material of the subject has only been partially implemented. There are multiple pending additions and changes, that have raised questions we do not have good solutions for yet, but with more research and experimentation they should be able to be integrated into the current system.

For example while the semantics of data channels are originally described using assignments as well, but their actual behavior is sort of implicitly more constrained, they cannot be assigned arbitrary values, their operations act more like methods on an object. The best solution for expressing these constraints is not clear yet, one possibility is to implement them as separate instructions instead of expressions.

Another case, which the original material does not explicitly discuss is the out of bounds indexing of arrays. As this is basically undefined behavior, our current implementation handles it by indexing them with natural numbers and returning a default value when the index is higher than the length, but a cleaner solution could be achieved by indexing them with the appropriate finite set for their length. This would require changing the current representation of arrays:
\begin{code}
    \>[0]\AgdaFunction{evaluateType}\AgdaSpace{}%
    \AgdaSymbol{(}\AgdaInductiveConstructor{Array}\AgdaSpace{}%
    \AgdaBound{A}\AgdaSymbol{)}\AgdaSpace{}%
    \AgdaSymbol{=}\AgdaSpace{}%
    \AgdaDatatype{ℕ}\AgdaSpace{}%
    \AgdaOperator{\AgdaFunction{×}}\AgdaSpace{}%
    \AgdaSymbol{(}\AgdaDatatype{ℕ}\AgdaSpace{}%
    \AgdaSymbol{→}\AgdaSpace{}%
    \AgdaSymbol{(}\AgdaFunction{evaluateType}\AgdaSpace{}%
    \AgdaBound{A}\AgdaSymbol{))}\<%
\end{code}

A dependent pair would be sufficient, where the second function's domain is controlled by the first element representing the length:
\begin{code}
    \>[0]\AgdaFunction{evaluateType}\AgdaSpace{}%
    \AgdaSymbol{(}\AgdaInductiveConstructor{Array}\AgdaSpace{}%
    \AgdaBound{A}\AgdaSymbol{)}\AgdaSpace{}%
    \AgdaSymbol{=}\AgdaSpace{}%
    \AgdaRecord{Σ}\AgdaSpace{}%
    \AgdaFunction{ℕ}\AgdaSpace{}%
    \AgdaSymbol{(λ}\AgdaSpace{}%
    \AgdaBound{l}\AgdaSpace{}%
    \AgdaSymbol{→}\AgdaSpace{}%
    \AgdaSymbol{(}\AgdaDatatype{Fin}\AgdaSpace{}%
    \AgdaBound{l}\AgdaSpace{}%
    \AgdaSymbol{→}\AgdaSpace{}%
    \AgdaSymbol{(}\AgdaFunction{evaluateType}\AgdaSpace{}%
    \AgdaBound{A}\AgdaSymbol{)))}\<%
\end{code}

The problem raised by this change is that we would need to ensure that all indexing done during the execution of the program stays within the bounds of the arrays, which is hard to enforce, since indexes can not only be given by constants during the construction of a program, but also dynamically by evaluating expressions, over the results of which we have little to no control.

Some other constructions, like intersection of other programs, which we did not have enough time to address are still missing as well, these could be added and the theorems based on them could be proven.

\subsection{Optimization for education}

Since our long term goals include the potential introduction of this formalization as a tool for teaching the subject, we aim to create ways in which it can be suitable for such a purpose. A specific type of task that is taught in practical lessons and serves as an exercise during evaluations is a problem, where the students are given certain parametric statements, the truth of which are assumed, and from those they have to decide if the validity of another statement can be proven, and if so, they have to give a deduction.

The definition of such tasks can already be performed in our current system, for example by giving a function with the function signature describing the problem, with its parameters being the assumptions and its result type representing the statement in question. If the body of the function, and thus the deduction can be constructed, it means that the implication holds. This could be utilized for either providing opportunity for the students to practice on their own with their computers providing them interactive help and being able to check the correctness of their own work, or if it is successfully introduced in classes, it may even be suitable for examinations with automatic grading.

The raw proofs for theorems and program properties are currently probably too complicated to be easily explained, but we have already started developing several helper functions that hide the lower level implementation and aim to provide a higher level interface, that could be used by students without having to understand all the underlying details.

\subsection{Automation}
Agda in itself already provides quite a good experience for interactively developing proofs using holes placed into the code and giving information about the types that are needed to fill them, but its auto solver is fairly limited to simple cases where the goal can easily be filled with an expression generated from the variables available in the close context. By expanding its search using the reflection capabilities of the language with methods specific to these tasks, more arguments could potentially be made implicit, and the verbosity of the otherwise sometimes inconveniently long proofs could be reduced.

\subsection{Generating executable code}
Another possible addition to the current system would be a way to export formal programs into a format that can either be further compiled or executed directly in an interpreter. This way it would be possible to experiment with the programs after proving their adherence to the given specification.

\chapter{Acknowledgements}

\section{Personal}
{\Large
    I would like to express my gratitude towards my two supervisors, namely Melinda Tóth and Ambrus Kaposi, who have both motivated me towards this project, provided the necessary background knowledge and were available when I was asking for advice, even under unusual circumstances.

    The constructive criticism of the manuscript by Mária Donkó and all the other help from others are also whole-heartedly appreciated.
}

\section{EFOP}
{\Large
    The author was supported by the European Union, co-financed by the European Social Fund (EFOP-3.6.3-VEKOP-16-2017-00002).
}
%%%%%%%%%%%%%%%%%%%%%%%%%%%%%%
%% Bibliography starts here %%

\newpage
\addcontentsline{toc}{chapter}{Bibliography}

%% There's more than one way to keep track of your citations.

%% For simply listing the citations by text you can use the thebibliography 
%% environment. See biblio.tex for an example. Comment out the following line
%% to use this style.
  
% \input{biblio.tex}

%% Another way is to use bibtex. The following command will process and 
%% include the citations listed in biblio.bib. The advantage of bibtex is that
%% you can simply copy-paste citations if the authors provided a bib-citation. 
%% For examples of such bib-citations, click the small "bib" link beside the 
%% articles at  https://plc.inf.elte.hu/erlang/refactorerl-academic-results.html

% \bibliography{biblio.bib}{}
\bibliography{references.bib}{}
\bibliographystyle{unsrt}

%%%%%%%%%%%%%%%%%%%%%%%%%%%%
%% Appendices starts here %%

% \addtocontents{toc}{\setcounter{tocdepth}{0}}
% \setcounter{secnumdepth}{3}
\addtocontents{toc}{\protect\setcounter{tocdepth}{0}}
\begin{appendices}

\chapter{Chapter in the appendices}
\label{apx:example}

The text of this appendix. 

You can refer to this appendix by using the label defined after the chapter directive (see \texttt{appendix.tex}).

\restoregeometry
\newpage

\chapter{Another appendix}
The text of this appendix.
\end{appendices}

\end{document}

% TODO: Agda, Formalization => Magyarázat, type theory -> logikai összekötők
% DONE? Cikk -> related works, Cite Fothi matemathical approach to programming
